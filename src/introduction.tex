\section{はじめに}
\label{sec:start}

情報処理学会では,基幹論文誌として論文誌ジャーナルの発行を行っている.こ
れまで論文誌ジャーナル編集委員会では,論文誌ジャーナルの論文掲載時のフォー
マットとしてA4横型2段組を採用してきたが,会員からの多くの要望に基づき,
A4縦型2段組に変更することにした.また,これまでは投稿時と掲載時の形式が
異なっていたが,今回のフォーマット変更に合わせて,投稿時も掲載時と同様の
A4縦型2段組で受け付けることにした.


これに伴い,\LaTeX のスタイルファイルも新しいものに変更した.本稿では,
まずそのスタイルファイルを用いた論文のフォーマットに関して述べる.新たな
スタイルファイルでは,極力特別なコマンドは使わずに,標準的な \LaTeX のス
タイルを踏襲している.論文フォーマットに関しては,\ref{sec:format}~章で
後述する指針に従って頂くが,そこに規定されていること以外は標準的な\LaTeX
のコマンドをそのまま使うことができる.本稿は,そのスタイルファイルを実際
に使っているので,論文執筆の際に参考にされたい.


\footnotetext{本文は実際には論文誌ジャーナル編集委員会で作成したものを
EC2017実行委員が一部変更したものである.}

また,論文誌ジャーナル編集委員会では,論文の執筆する際に,著者がするべき
こと,するべきでないことを「べからず集」としてまとめた.本稿の後半に,論
文の内容に関する指針になるように,「べからず集」の内容をチェックリストと
してつけているので,投稿する前の内容のチェックに利用されたい.
